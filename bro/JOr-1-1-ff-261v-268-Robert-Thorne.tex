\documentclass[a4paper,12pt]{article}
\usepackage{anysize}
\usepackage{hyperref}
\usepackage{fancyhdr}
\marginsize{4cm}{4cm}{1cm}{1cm}
\setlength{\parskip}{10pt}
\setlength\parindent{0pt}
\def\uuid{b3191902-7c65-4842-aae9-a37a2dfca6dc}
\def\authorname{Mike Jones}
\def\authoremail{mike.a.jones@me.com}
\def\shorttitle{Will of Robert Thorne, 10 October 1532}
\def\abstract{The last will and testament of Robert Thorne (d. 1532) recorded in the first volume of the Great Orphan Book. The copy in the Great Orphan Book is interesting because it includes the ordinances devised in 1545, during the mayoralty of Robert's brother, Nicholas, for the management of two of the bequests in the will. Namely, the interest free loan for young men wishing to undertake cloth working and the purchasing of grain and wood for the benefit of the town.}
\def\pubdate{April 2014}
\def\archivename{Bristol Record Office}
\def\archiveabbr{BRO}
\def\archiverefno{JOr/1/1, ff. 261v-268}
\def\archivedoctitle{Great Orphan Book, Vol. I}
\hypersetup{
	pdfinfo={
	    Author={\authorname},
	    AuthorEmail={\authoremail},
	    Title={\shorttitle},
	    Subject={\abstract},
	    Keywords={Thorne, Probate, Bristol, Great Orphan Book},
		UUID={\uuid},
		PubDate={\pubdate},
		Language={English, Latin},
		SubjectYear={1532},
		SubjectDate={10 October 1532},
		Archive={\archivename},
		ArchiveAbbr={\archiveabbr},
		ArchiveRefNo={\archiverefno},
		ArchiveDocTitle={\archivedoctitle},
		License={http://creativecommons.org/licenses/by/4.0/}
	}
}
\pagestyle{fancy}
\fancyhf{}
\rhead{\thepage}
\lfoot{http://bristollia.org/t/\uuid/}
\begin{document}
\title{\Large \shorttitle\\\normalsize \archivename \hspace{0 mm}, \archiverefno \hspace{1 mm} (\archivedoctitle)}\vspace{-5em}
\author{\small Edited by \authorname \hspace{0 mm} (\authoremail)}
\date{\small \pubdate}
\maketitle

\section*{Overview}
\abstract

Two other extant copies if the will (without the ordinances) also exist: the Prerogative Court of Canterbury, The National Archives, PROB 11/24/34; and in the Great Red Book, BRO 04719. The latter is printed The Great Red Book, ed. E. W. W. Veale, Vol. 3, pp. 124-30.

This transcription was made from the microfilm copy at the Bristol Record Office. Original punctuation and spelling has been retained and abbreviations are expanded in italics. Latin translations are provided on a best effort basis.
\section*{Text}

\textbf{[f.261v]} In dei nomine amen. The xvij$^{th}$ of May the yere of our Lord M$^{l}$ CCCCC xxxij. I Roberte Thorne beynge sekely but in my p\textit{er}fight mynde and reason, suche as it hathe plesed god to geve me fearyng dethe whiche is mortall for ev\textit{er}y leving creature  Willyng to dispose me the beste for the helthe of my soule That god will geve me grace to order this my Testament and last will, after the maner and forme folowing, ffurst I bequethe my Soule to almyghty god that created it and redemed it, desiryng of his infynyte m\textit{er}cy, to have m[er]cy on it, and put it in his glory that it was created for, to the laude and praise of his name. Also I belive in the blessed Trynytie ffather Son and holly Ghoost. And as all holly Churche holdethe and belyvethe as a ffaithefull \textit{Christ}yen man shulde doo, in whiche ffaithe I p\textit{ro}test to leve and dye, humbly beseching almyghtie god and our Lady Saint marye and all the Saynt\textit{es} of heven to be intercessors for me for all temptacions payne sekenesse or agony of dethe I may have grace so to persever in this holy belefe So that I maye dy in p\textit{er}fight ffaithe hope and Cheritie So that Soule may be saved. Also I will my body be buried in cristen buryall in suche Churche convenyent and that at my buriyng to be said Masse and dirige as a cristen man ought to have Also afore all thing\textit{es} I will my dett\textit{es} be paid, if any can be p\textit{rov}ed that I owe by Writting or recordes And also such Wares as to housholde stuffe, that is in my power of others be ever to their honors Of whiche wares I declare ij Chest\textit{es} geant velvett\textit{es} in the whiche ar 33 pec\textit{es} blacke app\textit{er}teynyng to Georgio Catanio more ij helles of perles m\textit{ar}ked w\textit{ith} a Spectacles, in the which is CC peerles that ar also the said Georgio Also I bequethe to the Childerne of my Sister Katheryne Woselay CCCC poundes sterling, that is to say ev\textit{er}y of them oon hundred poundes. And ev\textit{er}y oon to inherit of the other, if any dye afore lawfull age Also I bequethe to my Sister Alice Jackman and to hir childerne all that Thomas Jackman hir husbonde dothe rest owyng me at this p\textit{re}sent day whiche is ffive hundreth poundes and not certen dett\textit{es} that hathe p\textit{ro}ceded of Oyle and Sheepe whiche dett\textit{es} do app\textit{er}teyne to me and not hym, but oonly I bequethe the ready money that he owethe me for rest of accompte, to his Wife and Childerne to the S\textit{o}mme of the said CCCCC poundes litle more or lesse Also I bequethe to Edmunde Wythypoll ffiftye Poundes sterling And to Elizabethe Wythipoull his Sister other ffyftye poundes And to the Children of John Wythpoull of Mamisbury I bequethe ffyftye poundes to be rep\textit{ar}ted\footnote{i.e. distributed.} among them after the discrecion of my master Poule Wythipoll to be rep\textit{ar}ted more Orels lesse to ev\textit{er}y of them Also I bequethe to the Childerne of my uncle Willyam Thorne deceased named James Thorne and Thomas Thorne ffifty poundes apece and asmuche more to the ij doughters of the said Willyam Thorne the oon maried in the Citie the other in maydestone Also I bequethe towardes the making upp of the ffree Scole of Sainte Bartholomeus in Bristowe three hundreth poundes sterlyng and more that my Lorde Dalaware ouethe as by his Obligacion apperethe Also I bequethe CCC poundes to the Relief of the pore Com\textit{m}ons of Bristowe to be rep\textit{ar}ted after the discrecion of ij of the best men of concyens that may be founde in ev\textit{er}y p\textit{ar}isshe of the said Towne \textbf{[f. 262]} of Bristowe. Also I bequethe towarde the Redempcion of the ffeefarme and prisage of the saide Towne of Bristowe So it be redemed w\textit{ith}in iij yeres CC poundes Also I quethe to Agnes Amayne and to hir Sister there maried in bristowe XXX$^{li}$ to eche of them Also I bequethe to my godsonne Roberte Thorne the Son of my brother Nicholas Thorne C poundes Also I bequethe to Vyncent Thorne my Son beyng in Spayne iij$^{m}$ poundes sterlyng Which I will that Carlo Catanio that hathe the kepyng of hym at this p\textit{re}sent in Spayne shall reteyne of the goods of myne that the said Carlo Catanio and his brother hathe to the use \& benefitt of my said Childe till he be of lawfull age and setting it at Seynt Georges in Jeane. And if the said Vyncent my Son\textit{ne} dye afore he com\textit{m}e to lawfull age, the said iij M\textit{illesimo} $^{li}$ to retorne to my heirs Item I bequethe to Anagaria mother of the said Vyncent ffifty poundes w\textit{ith} condicion that she renounce all that p\textit{re}tence of inherytaunce of the bequest of hir said Son. Also I bequethe to the V almes houses in Bristowe C poundes to the relief of the pore people of the said Almes houses to be rep\textit{ar}ted after the discrecion of myne Executors Also I bequethe to Thomas Lucar and Willyam Ballarde Ffranceys Ffowler and Thomas Cornell, John Shipman, John Messam, Thomas Tyson, Humfrey Coston, Wilyam Pyckeryng, Willyam Harper, John Woseley to ev\textit{er}y of them X poundes And to Emanuell Lucar L poundes Also I bequethe to Thomas Moffet master in Gramm\textit{ar} Scole in Bristowe XXV poundes and to Robert Moffet his Son\textit{ne} x poundes Also I bequethe CCCCC poundes to be rep\textit{ar}ted in the Citie of London in the xxv$^{ti}$ Wardes xx $^{li}$ in ev\textit{er}y Warde among the pore householders after the discrecion of ij honest men chosen by my Executors in ev\textit{er}y warde of whiche is p\textit{ar}te alredy delyv\textit{er}ed, by Manuell Lucar to my Maister Poule Wythypoull Also of certeyn dett\textit{es} that arr Owing me by my Booke that is xiiij$^{li}$ vj$^{s}$ viij$^{d}$ that my Maister Poule Wythypoule owethe me And xxv poundes that Willyam Cyott of Bristowe owethe me, x poundes that Rafe Onley owethe me, iiij$^{li}$ xv$^{s}$ that Willyam Pepwall owethe me And x$^{li}$ iiij$^{s}$ ij$^{d}$ that Benet Jaye owethe me And xx$^{ti}$ poundes that Poull Wythipoll my maister owethe me. All thise forsaid dett\textit{es} I forgeve And bequethe it to ev\textit{er}y and wuld not it be axed Also I bequethe to my Sister Alice late wife of Thomas Jackman for hir ij Childerne C poundes a pece beside that bequest beforewrytten Also I wull there be delyv\textit{er}ed to Manuell Lucar C poundes sterlyng in money for to occupy for his owne behofe\footnote{belalf?} for the space of v yeres upon his owne obligacion w\textit{ith}out any Suerties. Also I bequethe towardes the relief of the prisoners L poundes at the discrecion of myne Executors w\textit{ith}in this Citie of London. Also I bequethe L poundes towardes towardes [\textit{sic}] the mariage a pore Maydens in London to be rep[ar]ted after the discrecion of my Executors Also I bequethe to the relief of pore prysoners in Bristowe L poundes to be rep\textit{art}ed after the discrecion of my Executors Also I bequethe towardes the mariages of pore maydens in Bristow L$^{li}$ to be rep\textit{ar}ted after the discrecion of my Executors Also I bequethe to the makyng of a place for \textit{mer}chaunt\textit{es} in the Strete where it shalbe thought by my Executors A C poundes w\textit{ith} Condicion that it be made w\textit{ith}in the space of iij yeres after my deceace Orels the said money to be disposed \textbf{[f. 262v]} after the will of my executors Also I geve and bequethe to the rep\textit{er}acion of highe ways from Comer Marshe to Bristowe C poundes at suche place\textit{s} as it shall seme best at the discrecion of my Executors Also I bequethe towardes the rep\textit{ar}acyons Abowte Bristowe of the highe ways C poundes. Also I bequethe to Aldermary Churche in Watlyng Strete\footnote{St Mary Aldermary, Watling Street, City of London.} x poundes Also I bequethe to Saint Nicholas Churche of bristowe xx$^{ti}$ poundes Also I bequethe to the iiij orders ffryers in bristowe to ev\textit{er}y order xx$^{ti}$ poundes the oon halfe to the Rep\textit{er}acion of the Churche and houses And the other halfe for their Sustentacion Also I bequethe to the iiij Prysons aboute London that is to saye Newgate Ludgate King\textit{es} Benche and M\textit{ar}shallsee C poundes to be delyv\textit{er}ed by myne Executors A noble ev\textit{er}y quarter of a yere to ev\textit{er}y house in bredde by myne Executors till the said C poundes be confirmed and paid. Also I bequethe to the pryson\textit{er}s of Bristowe C poundes to be bestowed in lyke maner as this in London. Also I bequethe CCC poundes in ready money, which I wulde have disposed in the Chambre of Bristowe to thentent that yerely may be made p\textit{ro}vision of Corne and Wudde for the Succor of the pore Com\textit{m}ons as to be bought as muche Corne and Wudde as that amounteth to in tyme of best Chepe, and to be delyvered unto them at that Price in tyme of best chepe, when it is wurthe more. So that always the said CCC poundes do contynue in the saide Chambre Also I bequethe that I wulde have deposed in the said Chambre of Bristowe CCCCC Poundes tothent to succour yong men that ar mynded to Clothemakyng in that Towne So that he that wulde bynde hymselfe and geve best Suertie and make in dede moost Clothe shall enyoye moost money w\textit{ith}out paying any intresses. But that he shall retourne the money that he receavethe into the Chambre at thende of xij monethes after such forme as my Executors shall devise Also I wull that there be none other accompt\textit{es} nor no other thyng be demaunded of Manuell Lucar my s\textit{er}v\textit{a}nt but that to be my Executor And that they and all others be contented of all that hathe past betwyne us And to receve thaccompte As by our bookes apperethe w\textit{ith}out any furder serche made against hym. Also I bequethe to Collyng\textit{es} Wife of this Citie whiche is doughter to Willyam Thorne my Cosyn xx$^{ti}$ Poundes Also I name for myne Executors whome I geve Charge to fulfill \& complete all that in this my said Testament Poule Wythypoule my maister for oon, Executor Emanuell Lucar as a nother Executor And Harry Hubberthorne ov\textit{er}seer This my Testament fulfilled my brother Nicholas Thorne to inheryt all the rest of my goods. Whome I make full Inherytour of all the rest of my goods Desiring hym afore all thing\textit{es} to see my testament fulfilled in good dedes of m\textit{er}cy that he will specyally rep\textit{ar}te it in good dedes of mercy to the relyef of the Comons of Bristowe to the pore people CCCCC Poundes above my bequest\textit{es} aforesaid Also I will that there be in the pouer of the said my Executors oon Thowsande Poundes to be distributed and ordered as my Executors shall seme best for my Soule Also that my brother Nicholas Thorne geve A iust accompte to myne Executors of all thay he shall rest owing to me w\textit{ith} payment Also I bequethe to my Sister Alice Jackman for hir selfe \textbf{[f. 263]} CCCCC m[ar]k[es] Also I bequethe to mye Executors and Ov\textit{er}seer in complyng this my Testament for the Confidence and trust that I have in them xx poundes Apece The residue of all my goodes aswell onthisside the See as beyonde the see this my p\textit{re}sent Will fullfilled I geve and bequethe to Nicholas Thorne my brother Also I bequethe to Willyam Wythypoull besides that, that is before bequethed xx$^{ti}$ poundes sterling And I in witnesse of trouthe the saide Robert Thorne have subscribed iy name the day and tyme abovewtyn And the forsaid Legacies to be paide according as money shall growe of my dett\textit{es} and wares to be solde./

Records present at the Subscribing and sealing of this last will of the saide Roberte Thorne the xviij$^{th}$ day of Maye An\textit{no} Do\textit{mi}ni MV$^{C}$XXXij I Richarde Reiguolde m\textit{er}cer Thomas Howson Clothewurker Willyam Macham Clothewurker and Edwarde Bawne Clothewurker.

\begin{sloppypar}Probatum fuit suprascriptum Testamentu\textit{m} coram Magi\textit{str}o Ric\textit{ard}o Gwent decretorum doctore prerogative tituli \textit{Christ}i Cant\textit{uariensis} Custode venerabiliu\textit{m} religiosoru\textit{m} q\textit{ue} viror\textit{um} prioris et Capituli eiusdem ad quos etc. Apud London decimo die mensis Octobris Anno d\textit{omi}ni mill\textit{es}imo quingentesimo tricesimo secundo ac approbatu\textit{m} et insinuatu\textit{m} etc. Et Com\textit{m}issa fuit administracio omni\textit{um} et sing\textit{u}lor\textit{um} bonor\textit{um} iur\textit{um} et credit\textit{orum} dict\textit{i} defunct\textit{i} Emanueli Lucar executori in h\textit{uius}mo\textit{d}i testamento no\textit{m}i\textit{na}to de bene etc. Ac de pleno et fideli inventario conficiend\textit{o} etc. necnon de pleno et vero Compoto redend\textit{o} etc. ad sancta dei Evangelia iurat\textit{i}. Paulo Wythypoull etiam executori in \textit{uius}mo\textit{d}i testamento no\textit{m}i\textit{na}to onus execuc\textit{i}on\textit{i}s eiusdem in se assumere expresse refutatu\textit{m} etc.\end{sloppypar}

[The above written will was proved before Master Richard Gwent, Doctor of Decrees, head of prerogative Christ Canterbury, custodian of the venerable religous men and prior to the chapter of the same at which etc. In London the tenth day of the month of October one thousand five hunderd and thirty two and approved and recorded etc. And the administration was granted of all and every goods, rights and credits of the said deceased to Emanuel Lucar, executor, in the same will named to well etc. And to prepare a full and faithful inventory etc. And also to render a full and true account etc. And sworn on the holy Gospels. For Paul Wythypoull also executor in the same will named expressly refused to take the burden of the same on himself.]

This bill the iij$^{de}$ day of November in the xxv$^{th}$ yere of the Reigne of our sov\textit{er}aign Lord King Henry the eight\footnote{3 November 1533} Witnessethe that Willyam Shipman Mayor of the Towne of Bristowe and the Comynaltie of the same Towne have receved of Emanuell Lucar Executor of the last will and testament of Roberte Thorne late of London m\textit{er}ch\textit{a}nttaillo\textit{r} deceased. CCCCC Poundes of good and lawful money of Englonde to and for to be deposited into the Chambre of the said Towne of Bristowe to and for the exercise use and entente to succour yong Men that ar mynded to make Clothe in the same Towne according to the last will and testament of the said Roberte Thorne Of which CCCCC Poundes and of ev\textit{er}y p\textit{ar}cell therof wee the said Mayor and Com\textit{m}ynaltie of the said Towne of Bristoll clerely discharge and acquyte the said Emanuell Lucar and the executors of the said Roberte Thorne their heirs and Executors forev\textit{er} And more \textbf{[f. 263v]} we do bynde unto to use the said Money According to the Last will of the said Roberte Thorne In Witnesse whereof the said mayor and Comynaltie to thise p\textit{re}ent\textit{es} hath put their Comon Seale the day and yere abovesaid. /

Be hitt knowen to all men by thise presents that wee Wyllyam Shipman mayor of the Towne of bristowe and the burgeises and Comynaltie of the same have receved the day of the making herof of Emanuell Lucar Executor of the Testament and last will of Roberte Thorne late Citizen m\textit{er}chu\textit{a}nttaillo\textit{r} of London by thandes of Nicholas Thorne brother to the said Robert CCC Poundes in ready money Whiche the said Roberte by his Testament aforsaide willed to be deposited in the Chambre of Bristowe to thentent that yerely may be made p\textit{ro}vision of Corne and Wudde for the socore of the poure Com\textit{m}ons of the same Towne Accordyng to the Will true meanyng and ?po\textit{r}porte of the said Testament of the fornamed Roberte Thorne Of Whiche CCC Poundes wee the said mayor burgeises and Comynaltie knowlage our selves to be well and trulie contented paid and satisfied And therof we clerely discharge Ande acquyte the fornamed Emanuell his heirs and Executors And ev\textit{er}y of theym by thise p\textit{re}sente Sealed w\textit{ith} the Comon Seale of the said Towne of Bristowe yoven in the yeldehall of the same Towne the xviij$^{th}$ day of ffebruary In the xxv$^{th}$ yere of the Reigne of Kinge Henry the Eight./\footnote{18 February 1534}

\textbf{[f. 264]}

The Ordinaunce devised made and establisshed the xij$^{th}$ day of July in the xxxvij$^{th}$ yere of the Reigne of our sov\textit{er}aigne lorde Henry the eight\footnote{12 July 1545} by the grace of god king of Englonde ffrance and Irelonde Defender of the ffaithe and of the Churche of Englonde and also of Irlonde in earthe sup\textit{re}me hedde by the whole assente consente and full agrement of Nicholas Thorne then beyng Mayor of the Towne or Citie of Bristowe and of all his bretherne / Burgeises of the Com\textit{m}on Councell then beynge present and assembled in the comon Councell house of the yeldehall [in the saide] Towne or Citie of and for the devysyng orderyng disposyng and settynge furthe of CCCCC Poundes sterlyng geven by Roberte Thorne late of the Citie of London marchaunttailor deceaced for to be occupied oonly in makyng of brode Clothes w\textit{ith}in this Towne or Citie of Bristowe for the relyef Welthe and Comforte of the Burgeises and Craft\textit{es} Men of the same Towne or Citie as herafter in articles folowethe. /

In primis that M$^{r}$ Mayor for the tyme beynge w\textit{ith} ffoure more of his bretherne and their Successors suche as by moost voyces of the whole Com\textit{m}on Councell of the said Towne shalbe elected named And chosen and ioyned w\textit{ith} the said Mayor Wherof the said Nicholas Thorne duryng his lyfe to be oon of them Whiche beyng ioyned w\textit{ith} the said mayor and bretherne shall have, from tyme to tyme, the orderyng disposyng and settyng furthe of the said CCCCC poundes to suche person and p\textit{er}sons and in suche porcyons, as by their good Wysedomes and discrecyons shalbe thought moost metest and best for the com\textit{m}on welthe of the said Towne And that in the settyng furthe of the same, they shall obs\textit{er}ve and kepe the rules \& constitucyons specified declared and mencyoned in the Articles of this p\textit{re}sent Ordynaunce. /

It\textit{e}m he that Offerethe hymselfe to make moost quantite of Broode Clothes w\textit{ith} suche money as he shall receve to hym shalbe made furste delyv\textit{er}ance of the said Money. /

It\textit{e}m that noo oon man or twoo men ioyned in oon Companye shall have above the Su\textit{m}me of xl poundes sterlyng of the said money But under that Su\textit{m}me at the discrecyon and libertie of the said Mayor and fouer bretherne for the tyme beynge. /

It\textit{e}m that ev\textit{er}y p\textit{er}son whiche shall receve and have any p\textit{ar}te or p\textit{ar}cell of the said money shall have two sufficient Sewrtyes w\textit{ith} hym suche \textbf{[f. 264v]} as ev\textit{er}y of them be ... ?truely .... the oon hundred Poundes sterlyng whiche wilbe bounde wi\textit{th} hym in doble the Sum\textit{m}e of Money which he shall receve by Obligacion w\textit{ith} Condicion therunto indorsed and in this Ordyn\textit{a}nce comprised exp\textit{re}ssed for the Payment of the saide Money And other Condicions to obs\textit{er}ve p\textit{er}forme and kepe a ... more playnly by the same it dothe appere according to the tenor and effecte therof.

It\textit{e}m that noo Parson recevyng any parte of the said money, shalbe taken Sewrtie oon for a nother. Or he that is Sewrtie for oon shall not be Sewrtie for an other. /

It\textit{e}m he that recevethe any competent Sum\textit{m}e of the said money shall make yerly for ev\textit{er}y L$^{s}$ sterlyng, so by hym receved, oon broode clothe at the least, of good drapery, conteynyng xxiiij yardes in lengthe and Seven quarters in breadethe from the water. And the same Clothe to be wurthe iij$^{li}$ vj$^{s}$ viij$^{d}$ sterlynge, or above. And that ev\textit{er}y suche Clothe shall have his Own\textit{er}s and wev\textit{er}s markes upon the endes of the same Clothe And more to have the lengthe that ev\textit{er}y Clothe holdethe at the water m\textit{ar}ked in leade by the Maister of the Tuckers for the tyme being w\textit{ith} a Seale of Iron engraved for the same. / 

It\textit{e}m ev\textit{er}y p\textit{er}son makyng Clothe w\textit{ith} the said money shall put it to be made oonlie to the wev\textit{er}s Tuckers Dyers and Sherman Inhabytu\textit{a}nt\textit{es} w\textit{ith}in this Towne of Bristowe And to their Sarvu\textit{a}nt\textit{es}, And not ells where. And for their wurkemanshipp and labors they shall pay them ready money. And no wares at any tyme. /

It\textit{e}m if any man receaving any Su\textit{m}me of the said money do make in any oon yere more quantitie in nombre of brode Clothes, then his Su\textit{m}me lymyted. That the ov\textit{er}plus shall not be allowed for any parte of the porcon of Clothes that he is bounde to make the yere following. /

It\textit{e}m that ev\textit{er}y p\textit{er}son receavyng any p\textit{ar}te of the said v$^{c}$ \textit{li} shall pay it agayne to the mayor and Chambreleyne for the tyme beyng the same day three yeres, then nexte ensuyng after the receipte therof. And that noo p\textit{er}son having any p\textit{ar}te of the said CCCCC poundes shall have it for any longer tyme then for three yeres at the fardest. Excepte the mayor and his said ffower bretherne for the tyme beyng, cannot fynde p\textit{er}sons more apte or able to make Clothe w\textit{ith}in this Towne of Bristowe And more p\textit{ro}fytable for the relife \& comforte of the Com\textit{m}on Welthe of the same Towne. /

It\textit{e}m that if any p\textit{er}son having or recevyng any p\textit{ar}te or porcyon of \textbf{[f. 265]} the said CCCCC Poundes do not truly make his Clothe or Clothes according to the forme tenor and effect\textit{es} of the Ordyn\textit{a}nce. Or do not make true payment of his whole Su\textit{m}me of Money at the day in his Specyaltie lymyted But that he shalbe compelled by the order of the lawe to pay it. That then he or they that so dothe disobey this ordyn\textit{a}nce or any p\textit{ar}te therof Shall notoonly fall into Danger and forfyture of the penalties specified and declared in his or their said Obligacion but also shabe excluded So that he shall nev\textit{er} after have any p\textit{art}e or p\textit{ar}cell of the said money. /

It\textit{e}m that all such Clothes as shalbe woven w\textit{ith}in the Towne, the wev\textit{er}s therof shall see, that at the furste ende of ev\textit{er}y Clothe, the true and iust nombre of Threedes and hundreds be noted and contayned in the same Clothe or Clothes, and lykewise the p\textit{ro}p\textit{er} m\textit{ar}ke of the same wev\textit{er}, w\textit{ith} also the marke of the Owner of the same Clothe And the same day or any other tyme after the Chayne of Chaynes Clothe or Clothes so putt and remaynyng on the Lome to be woven The said wev\textit{er}, before the discharging therof from the said Lome, shall geve knowlege to the maisters of the wev\textit{er}s for the tyme beyng, and cause them to co\textit{m}e and to p\textit{er}use and see the same Clothe or Clothes so being in the said Lome or Lomes, for the certificacion of the true and full content\textit{es} of the same. And if at any tyme defalte be founde in the said wev\textit{er} or wevers of for or upon any untrue certificat of the Threedes or hundreds noted in the ende of the same Clothe or Clothes. And also the p\textit{ro}per markes of the Wever and Owner, as is abovemencyoned. And that if he or they do not geve knowlege to the said M\textit{aste}r of Wev\textit{er}s for the tyme beyng accordyngly as before is specified  That then ev\textit{er}y of the said Wevers so founde in defalte to forfeyte ev[er]y tyme, -- iij$^{s}$ iiij$^{d}$ sterling  The oon halfe therof to remayne to the M\textit{aste}r of the said wev\textit{er}s. And the other halfe to the p\textit{ar}tie that makethe Complaynte therupon to the Mayor for the tyme beyng And the said M\textit{aste}r of the Wev\textit{er}s for the tyme beyng, to have for his paynes takyng theryn for ev\textit{er}y Clothe by hym so vewed seene and p\textit{er}used oon peny sterlyng, to be paide by the Wev\textit{er} of the said Clothe or Clothes ym\textit{m}edyatlye upon the vewyng therof. Whiche Peny shalbe repayed, by the Owner of the said Clothe or Clothes unto the said wev\textit{er} at the delyv\textit{er}ance or brynging whome of the same Clothe or Clothes. /

It\textit{e}m be it further Ordeyned that ev\textit{er}y Clothyer w\textit{ith}in this Cytie of Bristowe having any porcyon of the said Sum[m]e of Money shall p[re]pare and have in his house a Perche. On the whiche any Clothe or Clothes so woven as before is said / and brought whome They maye ym\textit{m}edyatlie drawe the same Clothe or Clothes ov\textit{er} and along the said Perche, to thentent and p\textit{ur}pose to app\textit{er}ceve and see, that true drapery be in the same made And wurkemanlie accomplysshed And also \textbf{[f. 265v]} That no ?Able be ?left owte And that noo ?Calles, Rouges, ?Waynstrok\textit{es} ?templestrookes, ?dowbleshott[es] or lyke defaltes befounde in any of the same Clothes and that no bryne or vryne, be ?co[m]myxte or putt into any colored Clothe that shalbe woven. And if any suche defalte be founde in any p\textit{ar}te or in all, and so dulie p\textit{re}sented. That then it shalbe ?leafull to and for the said Masters of the wev\textit{er}s to levey a ffyne, or for and upon ev\textit{er}y p\textit{er}son and p\textit{er}sons makyng or com\textit{m}yttyng any suche defalte or defaltes, according to their discrecyons. /

It\textit{e}m that ev\textit{er}y suche p\textit{er}son When his Clothe or Clothes be fully thycked and mylled, the said Clothe or Clothes, shalbe brought from the said Myll, unto a Place in Bristowe forsaid, called a packyng place And then and there the Maister of the Tuckers shall well and truly measure and mete ev\textit{er}y of the said Clothes so brought from the said Mill\textit{es} and the true contents therof to marke and enseale Or cause to be ensealed in Leade, at thende of the said Clothe or Clothes undre a sev\textit{er}all Seale for the same p\textit{ro}vided. And further shall vewe the myllyng and burlyng therof according to their olde Ordynary. And the same Maisters of tuckers to have for the true measuryng sealing and vewyng of ev\textit{er}y suche Clothe oon Peny sterlyng to be paid to them by the Tucker of the same Clothe Whiche Peny to be repaid to the same Tucker by the Owner of the said Clothe. And morov\textit{er} that no Tucker shall in settynge \& evenynge of of any Clothe at the Racke drawe or hale it above oon yarde in lengthe more then the iust Content\textit{es} of the same at Water, under payne to forfaytt for ev\textit{er}y defalt iij$^{s}$ iiij$^{d}$  The oon halfe to those of the M\textit{aste}r of the Tuckers for the tyme beynge  And the other halfe to the mayntenaunce of the said Tuckers hall. /

Provyded alweys that if any Tucker or Tuckers having any p\textit{ar}te or porcion of the said money for the true makyng of Drapery w\textit{ith}in the Towne of Bristowe shall from tyme to tyme ym\textit{m}edyatlye after the thickyng myllyng and bryngyng whome from the myll any of the said Clothe or Clothes of their owne makyng to their owne p\textit{ro}per uses. And before the evenyng and Rackyng of the same / shall call and cause the M\textit{aste}r of the Shermen for the tyme beyng to repare to the foresaid Packyng Place. Or any other place lymyted for the same, there to measure and trulie to mete the Clothes. Also of the p\textit{ro}per makyng of the said Tucker or Tuckers under their m\textit{ar}ke. And the true contents therof to marke and enseale Or cause to be marked \& ensealed in Ledde at the ende of the saide Clothe or Clothes Under a Sev\textit{er}all Seale ordeyned and appoynted for the indifferencye of the same Which Seale remaynethe in the handes of the said Maisters of Tuckers \textbf{[f. 266]} And the M\textit{aste}r of Sherman to have for his paynes for ev\textit{er}y Clothe so measured and marked oon Peny sterlynge. /

It\textit{e}m if the said maister of the Shermen at any tyme be founde in defalte for the true measuryng meatyng markyng and certefyng in Ledde the forsaid Clothes of the said accompte or any oon of them, As and when it shalbe requysite for ev\textit{er}y tyme so doyng he shall forfaytt iij$^{s}$ iiij$^{d}$, the oon halfe to be paid to the said Chambre And the halfe to the p\textit{ar}tie that fyndethe the defalte and complaynethe therupon to M\textit{aste}r mayor for the tyme beyng. /

It\textit{e}m it is farther ordeyned that if any P\textit{er}son or p\textit{er}sons havinge any Porcion of the said Money, at any tyme doo or shall sett to sale or utter any Clothe or Clothes beyng of the accompte aforsaid having not therunto affixed and sett the due and true Seale therfore p\textit{ro}vided in man\textit{ner} and forme aforesaid, they and ev[er]y of them so founde in defalte shall forfaitt for ev\textit{er}y suche defalte iij$^{s}$ iiij$^{d}$  The oon halfe to thuse of the Chambre aforesaid, And thother halfe, to the p\textit{ar}tie Whiche p\textit{re}sentethe the same unto the Mayor of bristowe for the tyme beynge. /

It\textit{e}m that no Sherman shall from tyme to tyme take into his house any man\textit{ner} or Clothe or Clothes from the Tucker And the same Clothe or Clothes put to his Shereborde to Shere. Onelesse it be well and sufficiently rowed before, that to be vewed and iudged by the ov\textit{er}sight and discrecion, of the M\textit{aster} of the Shermen for the tyme beyng. And the same M\textit{aste}r to be called to see the same by the Tucker upon payne to forfayte for ev\textit{er}y suche defalte, xx$^{d}$ sterlyng and the same M\textit{aste}r of Shermen to have for his Labor and payne oon Peny. And that no Tucker receve any suche Clothe or Clothes at any tyme from the Sherman. Wheryn any defalte of sheryng maye be founde

It\textit{e}m that ev\textit{er}y p\textit{er}son having any porcion or parte of the saide Money. When his Clothe or Clothes be made and fully wroughte shall bryng or cause to be broughte the same Clothe or Clothes ev\textit{er}y ffryday nexte after the makynge of the same to the backe hall at iij of the clocke at after none of the same daye. And then the M\textit{aste}r or keper of the same hall shall y\textit{m}medyatly enter into his Booke the nombre of the said Clothes w\textit{ith} the Leyngthes Colors and Value of the same And also the names of the Own\textit{er}s therof And the said M\textit{aste}r or Keper of the Backe hall shall have for entrying and regestrynge [\& sealynge]\footnote{inserted above the line} of ev\textit{er}y of the said Clothes in maner and forme abovesaide oon Peny sterlyng  Whiche Seale shalbe in Lead w\textit{ith} this marke TR

\textbf{[f. 266v]}

It\textit{e}m provided alweys that M\textit{aste}r mayor for the tyme beying w\textit{ith} thadvice consent and assent of his bretherne of the said Com\textit{m}on Councell may from tyme to tyme dymynyshe adde or alter any clause or clauses Article or articles mencioned in this Ordyn\textit{a}nce as they shall thynke expedyent and reasonable by their wisedomes and discrecions most necessary \& p\textit{ro}fitable for the com\textit{m}on Welthe of this said Towne So that the same dymynyshement\textit{es} addicions and alteracions be. / for the better makyng of true drapery w\textit{ith}in this Towne of Bristowe And also for the better gov\textit{er}naunce orderynge disposyng and settyng furthe of the said CCCCC Poundes And that noo parte noo p\textit{ar}cell of the same CCCCC Poundes be conv\textit{er}ted to any other use or uses, then for ?theucceace of Clothes or Clothe makyng w\textit{ith}in the saide towne of Bristowe./

The forme of the makyng of the Obligacion for the repayment of suche Sum\textit{m}es of money as shalbe lent owte of the said CCCCC Poundes ./

\begin{sloppypar}Nov\textit{er}int univ\textit{er}si p\textit{er} p\textit{re}sentes Nos. A. B etc. teneri et firmiter obligar\textit{i} C. D. Camerario Bristollie p\textit{re}d\textit{ic}te in \_\_\_\_ libris sterling Solvend\textit{is} eidem Cam[er]ario aut suo certo attornato vel Successoribus suis Cam\textit{er}ariis ip\textit{s}ius ville p\textit{ro} tempore existent\textit{e}. Ad quamquidem Soluc\textit{ionem} etc. Obligavimus nos etc.\end{sloppypar}

[Know all men by (these) presents we A. B. etc. Are held and firmly bound to C. D. Chamberlain of the aforesaid Bristol for \_\_\_\_ pounds sterling of money to the same Chamberlain or his certain attorney or his successors chamberlains of the same town for the time being. To making which payment etc. We bind etc.]

The Condicion of the p\textit{re}sent Obligacion is suche that if the above bounden. A. B. his Executors Admynystreators or assignes pay or doo to be paid to the maior of the said Towne or Citie of Bristowe and to the Chambrlayne of the same for the tyme beyng or to their Successors 
\_\_\_\_ li of good and laufull money of Englonde Upon the last day of the Monthe of May Whiche shalbe in the yere of our Lorde god, M$^{l}$CCCCCXL \_\_\_\_ And also do make or cause to be made w\textit{ith}in this Towne or Citie of Bristowe \_\_\_\_ Brode Clothes ev\textit{er}y yere during the space of iij yeres nexte com\textit{m}yng after the date herof And ev\textit{er}y of the said Clothes to be Wurthe at the leaste iij$^{li}$ vj$^{s}$ viij$^{d}$ sterling  And also do obs\textit{er}ve p\textit{er}forme falfill and kepe in the makyng and using of the said Clothes such ordre as is comprised in ordyn\textit{a}nce made and devised by the mayor Aldremen and com\textit{m}en Councell of the said Towne of Bristowe bering date the xij$^{th}$ day of Julye in the xxxvij$^{th}$ yere of the reigne of our sov\textit{er}aign Lorde King Henry the viij$^{th}$\footnote{12 July 1545} as in the same Ordyn\textit{a}nce more playnly it dothe appere That this p\textit{re}sent Obligacion to be utterly voyde Orels to stonde in his full strengthe power and vertue. /

\textbf{[f. 267]}

The ordinaunce devised, made and establyshed the xij$^{th}$ day of July in the xxxvij$^{th}$ yere of te regne of our sov\textit{er}eign Lorde Henry the eighte by the grace of god kyng of Englonde, ffrance and Irlende defendor of the ffaithe and of the Churche of Englonde and also of Irelonde in earthe sup\textit{re}me hedde by the assents consents and full agrements of Nicholas Thorne then beyng Mayor of the Towne or Citie of Bristowe and all the bretherne burgeyse of the Com\textit{m}on Councell then beyng p\textit{re}sent and assembled in the Com\textit{m}on Councell house w\textit{ith}in the Guyhall of the said Towne or Citie of and for the devysyng orderyng disposyng and settyng furthe aswell of CCXL Poundes sterlyng for the p\textit{ro}vision of Corne, As also of LX Poundes sterlyng for p\textit{ro}vision and to bye Wudde geven by Roberte Thorne late of the Citie of London m\textit{er}ch\textit{a}unt tayllor deceaced for to be sett furthe and ymployed w\textit{ith}in this Towne or Citie of Bristowe for the relief\textit{es} w[elthe] and Comforte of the burgeyses And Com\textit{m}ons of the same Towne or Citie as herafter folowithe. /

In primis at the day of the eleccyon of the Mayor and Shriffs shalbe chosen by the moost voyces of the com\textit{m}on Councell twoo able and sufficient men for Maisters of the Garnarde to whome shalbe delyv\textit{er}ed upon Mighelmas day nest aft\textit{er} by the way of Preste\footnote{An advanced payment.} XL Poundes of the CCXL poundes appoynted for the p\textit{ro}vision of Corne  That is to understonde to ev\textit{er}y of them, xx$^{ti}$ Poundes and more shalbe delyv\textit{er}ed to them at the same tyme, C poundes and so from tyme to tyme suche Sum\textit{m}es of Money to the value of CC poundes as by the mayor his bretherne And by the said twoo Maisters shalbe thought mooste necessarye and nedefull for to bestowe in Corne w\textit{ith} spede, and at suche tyme and place as it maybe thoughte beste chepe. And thay the said twoo Maisters from Mighelmas day forwarde have in their Garnarde suche sufficiencye and store of Corne, that the Com\textit{m}ons may have therof wekelye upon the m\textit{ar}kett days, that is to saye Wednesday ffrydaye and Satterdaye LX. LXXX. or more busshells of Corne, as they ?lyste or nede, at the selfe same Price that it cost, clerely putt into the Garnarde addyng therunto ij$^{d}$ upon ev\textit{er}y quarter solde to the bakers and to the Com\textit{m}ons Whiche the Maisters of the Garnarde shalbe allowed in their accompte ffor the biyng and sellyng good order and true measure geving of the saide Corne to the said Com[m]ons. /

\textbf{[f. 267v]}

It\textit{e}m that the said twoo Maisters of the Garnarde and ev\textit{er}y of them shall fynde twoo sufficient Sewrties at thende of viij dayes ym\textit{m}edyatly folowing the eleccyon of the mayor whiche shalbe bounde by Obligacion for to repaye unto the mayor Shriffs and Chambrleyn and to their Successors for the tyme beyng upon Mighelmas Eve nexte folowing then com\textit{m}e Twelvemonethe aswell the XL Poundes of Prest money as the other money receved to bye Corne And also then there that is to saye viij dayes after the Mayors eleccon, the said ij Maisters shall make and geveupp a true and iuste accompte of their receipt\textit{es} employment\textit{es} and payment\textit{es} to the mayor Shriff\textit{es} and Chambrleyne of Bristowe for the yere beyng at Whiche accompte shalbe alweys p\textit{re}sent Nicholas Thorne duryng his lyfe. And in their Payment shalbe allowed shuche quantitie of Corne as shall then remayne in the saide Garnarde unsolde by the reporte of the two newe maisters of the Garnarde or twoo other indifferent p\textit{er}sons therunto appoynted. /

It\textit{e}m if it chaunce the price of Corne do fall at any tyme w\textit{ith}in the yere. So that the Corne p\textit{ro}vyded by the Maisters of the Garnarde can not be solde at the price that it coste w\textit{ith} all charges. That then the same Corne w\textit{ith}in the tyme of twoo monethes nexte folowing or rather after the price so fallen to be delyv\textit{er}ed to the Bakers of the Towne of Bristowe by Mayors Com\textit{m}adement and they to bake it after suche valiacion and Sise as it cost. All charges \& cost\textit{es} accompted and to them allowed consideryng that it was bought for the com\textit{m}on Welthe of the saide Towne. /

It\textit{e}m that the mayor and his bretherne shall from yere to yere at the tyme above assigned admytte the twoo olde Maisters or oon of them to contynue in his or their Office for an other yere, Or to discontynewe them bothe or oon of them by their discrecyons, as they shall fynde or see cause And for them or hym so discontynued to put an other or others in his or their places or rombes suche a p\textit{er}son or p\textit{er}sons as they shall thynke mete and sufficient therfore And this twoo newe maisters of the Garnarde so admyttted and chosen and consequenthe all other s succedyng them in the said Office shall obs\textit{er}ve and kepe for their tyme the Order beforespecyfied. Also the olde Mayor shall from tyme to tyme contynually forev\textit{er} upon Mighelmas daye openly showe in the yeldhall to all the Com\textit{m}ons there assembled the said CCXL Poundes appoynted for the p\textit{ro}vision of Corne Accompting in the same Sum\textit{m}e the iuste value of the Corne that at the same season shall remayne on the Garnarde unsolde And then and there beyng p\textit{re}sent, the ij newe Maisters of the \textbf{[f. 268]} Garnarde chosen for the rulyng gov\textit{er}n\textit{a}ce and bestowyng of the said Money in Corne for the Com\textit{m}ons as they shall thynke best and moost beneficiall for the Comon Welthe of the said Towne. /

Ffurthermore it is ordeyned and determyned by the mayor and his bretherne that yerelye \_\_\_\_ dayes after Mighelmas shalbe elected by the moost voyces of the Comon Councell ij or iij honest men suche as by their discrecions shalbe thoughte to have best knowlege experyence and acquayntance in and for the biyng of Wudde and amonge them shalbe delyv\textit{er}ed the xxiiij$^{th}$ day of Marche next after folowing by the discrecion of the said mayor and his bretherne LX Poundes sterlyng at the receipte whereof ev\textit{er}y of them for his porcion And twoo sufficient Sewrties w\textit{ith} hym shalbe bounde by Obligacion in C li. that is to saye ev\textit{er}y of them accordyng to the money that they receve for to have Wynter and Somer in Pyles w\textit{ith}in the Towne at a place or places by the Mayor Lymyted such Store and plentye of breunyng\footnote{burning?} Wudde of all soortes, as may satisfye the Comons of this Towne of Bristowe for their owne use, at all reasonable callyng\textit{es} and requyryng\textit{es} for their money to be solde and delyv\textit{er}ed to them, at such Price reasonable, as ev\textit{er}y sorte of Wudde vewed and seene by the mayor Shriff\textit{es} Chambrelryne and certeyne more of the bretherne as shalbe by them reasonable prysed and valued.

Also that day Twelvemonthe that they receve the LX Poundes aforsaid ev\textit{er}y of them shall brynge in And geue to the mayor Shriffs \& Chambreleyne in the presence of the Councell a true and iuste accompte w\textit{ith} Payment of such Porcyon of money as they and ev\textit{er}y of them receved. In the Whiche accompte shalbe allowed for Payment in the mayors Prises suche quantitie of Wudde as by the reporte of twoo indifferent p\textit{er}sons therunto appoynted shall at that tyme reste in the Pyles aforesaide unsolde. /

Provided alweys that suche p\textit{er}sons as shall receve the said LX Poundes before the receipte therof shall fynde two sufficient Sewrties to be bounde w\textit{ith} them by Obligacion in double the Sum\textit{m}e that they do receve / for the true repaymente of the sames Sum\textit{m}es, that day twelvemothe. Orels to incurre the penalytie of the same. / 

\end{document}

