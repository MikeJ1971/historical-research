\documentclass[a4paper,12pt]{article}
\usepackage{anysize}
\usepackage{hyperref}
\usepackage{fancyhdr}
\usepackage{pdflscape}
\usepackage[T1]{fontenc}
\setlength{\headheight}{15pt}
\marginsize{2.5cm}{2.5cm}{1cm}{1cm}
\setlength{\parskip}{10pt}
\setlength\parindent{0pt}
\def\uuid{b070c4f3-6fa9-4592-b6eb-aad8c4070b7d}
\def\authorname{Mike Jones}
\def\authoremail{mike.a.jones@me.com}
\def\shorttitle{Will of John Collas, Merchant of Bristol, Gloucestershire, 19 December 1517.}
\def\abstract{The last will and testament of John Collas (d. 1517). The will was written on the 24 July 1517 and was proved on 19 December of the same year.}
\def\pubdate{July 2014}
\def\archivename{The National Archives}
\def\archiveabbr{TNA}
\def\archiverefno{PROB 11/19/18}
\def\archivedoctitle{Will of John Collas, Merchant of Bristol, Gloucestershire, 19 December 1517.}
\hypersetup{
	pdfinfo={
	    Author={\authorname},
	    AuthorEmail={\authoremail},
	    Title={\shorttitle},
	    Subject={\abstract},
	    Keywords={Collas, Probate, Bristol, Somerset, Ship},
		UUID={\uuid},
		PubDate={\pubdate},
		Language={English, Latin},
		SubjectYear={1517},
		SubjectDate={19 December 1517},
		Archive={\archivename},
		ArchiveAbbr={\archiveabbr},
		ArchiveRefNo={\archiverefno},
		ArchiveDocTitle={\archivedoctitle},
		License={http://creativecommons.org/licenses/by/4.0/}
	}
}
\begin{document}
\title{\Large \shorttitle\\\normalsize \vspace{1em} \archivename \hspace{0 mm}, \archiverefno\footnote{In the digital copies of wills it isn't always possible to ascertain the folio numbers used in the original bound volumes. The digital reference number is therefore given.} \hspace{1 mm}}\vspace{-5em}
\author{\small Edited by \authorname \hspace{0 mm} (\authoremail)}
\date{\small \pubdate}
\maketitle

\section*{Introduction}
\abstract

The will refers to a ship called the `Mary Towre' which Collas bequeaths to a son John (3/4 portion), a daughter Joan (1/8 portion) and his friend Adam Skelton (1/8 portion). Collas is recorded in the Bristol `particular' accounts of 1517 using the ship to import wine from Sanl\'ucar de Barrameda, 29th January, and to export corrupt wine to Ireland on 1st April and again on 9th May with wine, beans and cloth. Adam Skelton also shipped goods on the trips to Ireland.\footnote{Flavin, S. and Jones, E. (eds.), \textit{Bristol's trade with Ireland and the Continent, 1503-1601: The evidence of the Exchequer customs accounts, Bristol Record Society}, 61 (2009), pp. 125, 156 and 162. A ship called the `Mary Towre' is also recorded as arriving from Lisbon on 22 December 1503 (pp. 17-18) and departing for Lisbon on 26 April 1504 (pp. 66-7). A John Collas is listed on the former entry as a merchant shipping wine and wax to Lisbon.
}

\section*{Editorial Note}

The transcription was made from a digital image supplied by The National Archives. Original punctuation, spelling, lineation and indentation are preserved. Superscript characters are lowered and contractions expanded with supplied letters italicised. The thorn (\th\ or y) is replaced with `th' and a terminal graph with `es'. The ampersand brevigraph is silently expanded to `and' in English and `et' in Latin. Letters within square brackets are supplied by the editor. Engrossed hand are shown in bold text.

\begin{landscape}\section*{Text}
\textbf{In the name of god amen} The xxiiijti day of July In the yere\hfill Test\textit{amentu}m Joh\textit{ann}is\newline 
of our lord god ml vc xvij I John Collas of the noble towne of Bristowe m\textit{er}chaunt hole of \hfill Collas\footnote{The testament of John Collas}\newline 
mynde and parfite memory thanked be almighty god make and ordeyn my testament\newline 
conteynyng my last wille in this maner folowing ffirst I bequeth my soule to almighty\newline 
god and to his blissed moder saint mary and to all the celestiall company of hevyn. And\newline 
my body to be buried w\textit{ith}in the holy crowde of saint John Baptist my parishe Church. Also\newline 
I bequeth to the moder Church of Worcetor iij s iiij d. Also I geve to sir John Toste my Curat\newline 
for tithing\textit{es} and offering\textit{es} necligently forgotten and to pray for me xx s Item I geve and\newline 
bequeth to the goode werk\textit{es} of the said Church of saint John xx s / Also I geve and bequeth\newline 
all my londes and ten\textit{emen}t\textit{es} with their appurten\textit{a}nce w\textit{ith}in the towne of bristowe to Johane\footnote{Joan Collas died in 1519, see TNA PROB 11/19/244.} my\newline 
wife during the terme of hir lyfe / And after hir decesse to remayne to John my sonne and\newline 
his heires forevermore / Also I geve and graunt to the said Johane my wife / my maner of\newline 
harpisford\footnote{Hapsford in the parish of Great Elm, Somerset} w\textit{ith}in the parishe of Elme in the Countie of Somerset w\textit{ith} myll\textit{es} leses and pastures\newline 
and all their appurten\textit{a}nc\textit{es} during the terme of hir naturall lyfe to use them to hir most profite\newline 
and after hir decesse I geve and graunt the said Maner w\textit{ith} myll\textit{es} leses and pastures to Richard\newline 
Collas my sonne and to his heires laufully of his body begotten And for fawte of yssue to\newline 
Remayne to the next heires of my body laufully begotten. Also I geve and graunt my howse\newline 
in Elme aforsaid to Luce my suster during the terme of hir lyfe And after hir decesse I will\newline 
it remayne to Richard Collas my sonne in fourme and and maner above reherced / ffurthermore\newline 
I ordeyn and make Johane my wife and John Collas my sonne my true and laufull executors\newline 
ffirst to se that my body be honestly buried and especially to paye my dett\textit{es} that may playnly\newline
appere by obligacions bill\textit{es} scorys or any other laufull prove / than I woll that John Collas\newline 
my sonne shalhave thre part\textit{es} of my ship called the Mary Towre for his porcion of my good\textit{es}\newline 
And also I geve half a quarter of the said Ship to my doughter Johane Collas And the\newline 
other half quarter I geve to my frende Adam Skelton to praye for me And to use the said\newline 
Ship for the informacion of my forsaid sonne and doughter The Residue of all my goodes\newline 
meveable and unmeveable above not bequethed I geve and bequeth to Johane my Wife to\newline 
dispoase for the helthe of my soule as she shall see most to the pleasure of god And also I ordeyn\newline 
and make sir John Toste dean of Bristowe overseer of this my testament conteynyng my\newline 
last will to se that it be perfourmed effectually and also I wille that my wife shall reward\newline 
him for his labour after hir discrecion And I will also that this testament in the which I\newline 
have declared my last wille shall anull all other testament\textit{es} that I have made afore In\newline 
witnesse wherof to this I have putt my Seall Thise being witnesse Sir Will\textit{ia}m Brygtyn\newline 
sir Thomas Bocton Adam Skelton wither other moo -/



\textbf{Probatum} fuit testamentum suprascripti defuncti Coram do\textit{mi}no apud Lamehith\newline 
xix die mensis decembris Anno do\textit{mi}ni mill\textit{es}imo quingentesimo xvij Iurement\textit{o} Joanne Relicte et\newline 
Executrici in hui\textit{usm}o\textit{d}i test\textit{ament}o no\textit{m}i\textit{n}ate\textit{e} In p\textit{er}sona d\textit{omi}ni Roberti Colett cap\textit{ella}ni procuratoris in hac p\textit{ar}te Ac approbat\textit{um}\footnote{The testament of the above-written deceased was proved before the lord at Lambeth the 19th day of the month of December 1517 by the oath of Joan, widow and executrix named in the same testament, in the person of Sir Robert Colett, chaplain, proctor in this regard. And probated}
\end{landscape}

\end{document}